\chapter{项目创新点与意义}
\label{cha: significance}

\section{项目创新点}

在学术研究上,团队尝试应用自然语言处理技术解决实际生活中的法学问题。通过长时间的探索与交流,我们总结并提出了多个常见的法学任务,并将其形式化为能够利用自然语言处理模型进行处理的常见任务。并且在多个任务上,团队提出了创新性的模型,达到了不错的效果。

另一方面,通过与法律从业人员长时间的沟通交流,团队提出的任务都是建立在切实解决实际问题的基础之上的。我们致力于打造一个功能全面、技术可靠的案情分析平台。相比于市面上现有的一些法律产品,我们平台的功能更加全面、实用,结果更加可靠。

总结而言,项目有以下亮点与创新:

\begin{enumerate}[1)]
	\item 团队创新性地首次提出智慧司法领域的多个任务的规范化流程,并以此为基础构建了多个相应的数据集,促进了智慧司法领域学术发展。经过两年多不断的实践探索,团队创新性地首次提出案由预测、相关法条预测、刑期预测、关键词抽取、类案推荐等任务的规范化流程。以此为基础,团队构建并公开了大规模的司法判决预测数据集、类案推荐数据集、关键词检索数据集。这些数据集受到司法智能领域学者的广泛关注,成为了相关领域任务评测的benchmark。
	\item 在判决预测模块的三个子任务(案由预测、相关法条预测、刑期预测)上,团队首次提出要素式多任务学习模型,解决了传统深度学习模型在法律领域遇到的样本分布极度不均、相似罪名易混淆、预测结果互相矛盾的问题。模型通过将罪名拆分成基本案件构成要素,并利用子任务之间依赖关系,实现了预测效果的提升。实验结果表明,在真实场景数据集上,我们的模型实现了10-20\%的误差率降低。
	\item 在关键词预测模块,团队提出基于栅栏式记忆神经网络(Lattice-LSTM)的关键词抽取模型,解决了传统中文关键词抽取依赖中文分词导致错误累积,字级别预测模型信息不足的缺陷。在构建的真实场景数据集上,在以往算法的基础上,模型实现了7-10\%的预测效果提升。
	\item 团队在判决预测、关键词抽取的基础上,提出了语义相似的类案检索模块。传统搜索引擎大都基于词频统计,只能做到文本相似检索,无法满足相似案件推荐的需求。团队通过深度学习模型将文本映射到低维向量空间,利用向量相关程度衡量文本语义相似性。以此为基础,通过两次重排序实现了高效的相似案件检索,能够有效地帮助法官实现类案类判。
\end{enumerate}

\iffalse
\begin{enumerate}[1)]
	\item 尝试利用自然语言处理的最前沿的技术解决相关的法律问题。通过对案由预测、相关法条预测、刑期预测、关键词抽取、类案推荐等功能的实现,平台可以帮助法律领域从业人员减免重复工作,成为辅助其工作的好工具;同时也可以为大家身边遇到的法律问题提供的解决方案,成为无法学背景的非专业人士的好帮手。
	\item 提出了全面的法学基础任务,能够对每一段案情进行多角度全方位的分析,更好的满足了用户需求。通过与从业人员的多次的深入交流,我们获知了他们在工作中碰到最多的几大问题,并进行针对性的解决,真正做到了充分了解用户需求。
	\item 在判决预测模块,我们提出了一个能够捕捉子任务间依赖关系的多任务学习模型,超过了以往模型,实现了state-of-the-art效果。在我们提出的几个任务之间,往往有着很强的依赖关系,例如案由与法条之间具有很强的映射关系。模型通过捕捉这些子任务之间的映射关系,提升其效果。
	\item 首次在关键词抽取任务上运用并改进Lattice-LSTM模型。克服了词级别模型过度依赖于中文分词效果、字级别模型语义信息不足的缺点,在传统的序列标注模型上实现了大幅提升。
	\item 在类案检索模块,我们提出了一个基于关键词抽取、案件语义理解的模型,做到了在语义层面上的相似性检索。模型首先抽取案情关键词标签,通过标签缩小候选的相似文本集合,再进一步通过不同文章的文章向量之间的距离来衡量案情的相似程度。相比于传统搜索引擎的文本相似性检索,此搜索模块可以做到真正的语义相似性检索。
\end{enumerate}
\fi

\section{项目意义}
人工智能现已上升为国家战略,各行业各领域也纷纷响应并加入到这场人工智能的改革当中。法院作为保障社会公平公正的重要一环,与社会生活密切相关。把大数据、人工智能与司法体制改革结合起来,将会给司法工作注入前所未有的创造力。

现在全国3519个法院和9279个人民法庭通过专网已经实现了互联互通,各级法院以每分钟一次的频率向最高人民法院大数据管理和服务平台自动汇集新收集的各类案件数据,目前该平台已汇集了1亿多件案件数据和$6,000$万份法律文书。显然,人工智能介入司法领域是时代发展的必然趋势。当前案多人少的严峻局面并没有因司法改革的深入得到根本性转变。以2015年和2016年为例,全国各级法院审结一审刑事案件分别为$109.9$万件和$111.6$万件,分别比上年增加了$7.5\%$和$1.5\%$,有许多基层法院刑事法官年均结案数量已达200件以上。在员额法官增加受体制钳制的情形下,通过人工智能来提高审判工作效率是一个必然选择。

况且分析近十年依法纠正的三十余件重大冤假错案发生的原因,可发现既有事实不清,更有证据收集上的不规范、数量上的不充分、标准上的不一致,严重背离了“证据三性要求”和排除合理怀疑的定罪标准,当然也不排除其中人为意志因素的干扰。这就需要利用司法大数据对这些问题案件进行全面梳理以分析原因、发现问题、总结经验、制定标准和规则,并将证据收集要素嵌入人工智能模块,进而为司法人员合法有效收集证据提供确切指引,以避免以往的随意性或不规范性。显然,人工智能所体现出的智能性和高效性与司法人员的创造力结合起来构建人力和科技深度融合的司法运行新模式,这是以审判为中心诉讼制度改革的应有之义。

总而言之,项目的完成有着以下的意义:
\begin{itemize}
	\item 提升法官、律师工作效率。在我国目前案多人少的严峻形势下,平台的高效性、可靠性可以在保障审判质量的同时,大大提高法官、律师的工作效率。司法与人工智能融合的模式可以通过推动案件信息自动检索、法条案例自动推送将法官、律师从繁琐、重复度高的工作中解放出来,大幅度减轻法官量刑办案的工作量,有效缓解法院案多人少的突出矛盾。
	\item 促进司法公平公正。基于大数据的人工智能系统,可以排除判案上的人为意志因素的干扰。平台利用了超过百万的数据进行模型训练,发掘判案标准,可以为法官提供类案推送、裁判比对、数据分析等服务,对偏离度过大的案件自动预警、及时评查,推动“类案类判”,促进裁判尺度统一。
	\item 提高法律文书质量。平台可以用于法律文书智能分析,帮助司法人员发现和纠正事实证据遗漏、法条引用错误,有效提高文书质量。
	\item 提供及时的法律咨询服务。我国人均律师数量远低于发达国家,这非常不利于司法的普及。平台凭借简单的交互方式,全面准确的事实分析,可以很好的帮助到每一个人解决身边的法律问题。
\end{itemize}

\iffalse
\begin{itemize}
	\item 在专业人员数量受限制的情况下,平台可以在保障审判质量的同时,利用人工智能算法来大大提升审判工作的效率。
	\item 排除人为意志因素对审判结果的干扰,利用司法大数据对判案制定统一的标准。
	\item 成为每一个人的法律咨询助手,平台提供了详细、全面的案件案情分析,能够很好地为每一个人提供法律知识,帮助大家维护自身权益。
	\item 提供了人工智能与司法人员结合的判案新模式,这将是未来的发展新模式。
\end{itemize}
\fi



