\chapter{相关工作}
\label{cha:related}
当前人工智能与深度学习正在以势不可挡的冲击力影响着我们的生活,我们国家至关重要的司法领域并没有选择回避,而是采取了一个积极的姿态主动拥抱技术,其目的就是通过人工智能与法官审判制度的融合来促进司法改革。法律结合人工智能同样已经是大势所趋,法律智能这一领域也受到了学术界广泛关注,与法律智能有关的任务也层出不穷。但是,由于技术发展时间短,许多任务并未达到令人满意的效果。这也导致了目前法律智能产品虽多,却大都无法满足实际使用的需求。


% \section{学术研究现状}
\section{法律智能}

法律智能领域的研究已经持续开展了几十年,不同的学者从不同的角度提出了很多不同的法律智能任务。其中研究最为广泛的便是判决预测任务,在该任务中,算法模型需要通过阅读相关案情描述来给出判决结果,其中判决结果包括罪名、刑期、罚款等。

早在上个世纪就有许多学者尝试利用概率统计的数学模型来给出案件判决结果与事实描述词语之间的概率分布\cite{kort1957predicting,ulmer1963quantitative,segal1984predicting}。随着机器学习算法与文本挖掘算法在本世纪初的迅速发展,更多学者将判决预测抽象成一个文本分类任务,并通过手动抽取文本的浅层次特征来抽取文本信息\cite{liu2006exploring,lin2012exploiting,aletras2016predicting},并运用传统机器学习模型(朴素贝叶斯分类算法、决策树分类、支持向量机算法)来进行预测。但是这种传统做法需要耗费大量的精力总结文本规律,抽取浅层特征,并通过手动设计的规则进行预测,无法适应灵活多变的语言表示,因此并没有得到很好的运用。

到了近几年,深度学习技术高速发展,许多自然语言处理领域(NLP)中的问题都得到了很好的解决\cite{kim2014convolutional,khan2010review,tang2015document}。于是,许多学者又开始运用神经网络来结合法律知识来进行罪名的预测。例如,\citet{luo2017learning}运用NLP中常有的注意力机制,在预测过程中引入法条内容,提升预测效果;\citet{shen2018legal}利用记忆网络(Memory Network)来端到端训练相关法条预测与罪名预测任务。这些模型在相应的筛选后的测试集上取得了较好的效果,但是由于这些任务都对实际运用场景做了简化,并未考虑法官判决时碰到的实际问题,因此这些模型在实际应用中仍然无法满足需求。

总而言之,经过长时间学者们的不断努力,判决预测任务取得了较大的突破,但是几乎所有模型都对问题做出了相应的简化,导致目前判决预测任务依旧面临着很大的问题而无法落地使用。

% 法律智能领域的研究已经持续开展了几十年,早在上个世纪就有许多学者尝试用概率统计算法来对案件进行分析[1][2],通过数学模型来预测案件的判决结果。作为法律智能的基础任务,判决预测这一任务自此也受到了广泛关注,随着本世纪初机器学习算法的发展,学者们将这一任务归结为文本分类任务,并尝试通过人工抽取特征并利用传统机器学习算法(朴素贝叶斯分类、SVM算法)[3][4]预测判决结果,到了近两年,学者又将深度学习技术运用其中[5]。这些模型虽然在测试集中取得了较好的效果,但是因为对问题的简化以及未考虑到判案时的实际因素,均无法应用到实际生活中。

% 与此同时,法律智能上还有许多其他问题也受到学者们的广泛关注。例如


与此同时,受到计算机学者、法学学者共同关注的法律智能领域许多其他任务也被纷纷提出。希望能够帮助解决我们生活中的种种法学问题。例如给定案情描述判断相关法条,与判决结果预测类似,学者也尝试使用文本分类建模,利用神经网络强大的抽取特征能力,寻找最相关的法条。不仅如此,人们还尝试利用人为定义的规则构建一套系统来判断法律问题的正误,但是此类方法缺乏可拓展性,且需要耗费大量的人力物力。


总而言之,目前法律智能领域受到广大自然语言处理领域的学者的欢迎,然而因为计算机学者法律知识的相对缺乏,以及学术研究上对问题的简化,目前学术领域研究虽然取得了很大的进展,但是离技术的落地与实际应用仍有一段很长的距离。

% 通过长期的调研与讨论,我们团队提出了一个新的模型,利用了机器学习中的多任务学习机制,通过多个子任务的联合训练提升了各个任务的效果;同时,考虑到法官的判案经过,我们提出了一个完整的模拟法官真实判案过程的模型,超过了之前提出的baseline模型,达到了真实可用的效果。





% \section{类案检索}

% \section{商业产品现状}

% 综上所述,目前市面上的法律产品并没有使用人工智能领域的前沿技术,而其服务的受众群体也是极其有限的,无法律知识背景的人在使用这些工具时会遇到很大的困难。


\section{关键词抽取}
关键词抽取顾名思义就是对文章进行总结,通过从一篇文章中抽取一些比较重要的词语或词汇,帮助阅读者快速的理解文本的含义;而这些抽取出的关键词不仅仅可以辅助阅读,还可以在检索、数据挖掘等方面发挥巨大作用。尤其在如今的互联网环境下,每天都会产生海量的数字化文本,能高效有效的从这些数据中挖掘信息,离不开关键词抽取的辅助。

在自然语言处理领域,关键词抽取也是一个经典传统的问题,经过几十年以来的研究诞生了一个个经典的算法,其中最具有代表性的也是目前应用最广的算法有TF-IDF、TextRank、Rake等。
TF-IDF算法的主要思想来自于对已有语料的统计,它认为词语的重要性与词语在文档中的出现次数呈正比,但会随着这个词语在语料库中的出现频率而降低。通常一个词语在文档中多次出现,说明这个词语在文档中是有意义的,但如果这个词在其它文档中也经常出现,则说明这个词其实并没有那么重要。换句话说一个词语如果在某篇文章中出现次数多,但在其它文档中没有怎么出现过,那么这个词语很有可能就是这篇文档的关键词。而TF-IDF算法对于重要性的计算,就是基于这种思路,使用逆文档频率乘以词语在当前文档出现的频率作为重要性的度量。
与TF-IDF不同,TextRank算法的主要思想则是通过对数据进行建模来获取词句的重要度,将整个文档建模为一个图,文档中每一个句子建模为图中的一个顶点。它的前身是大名鼎鼎的网页页面排序算法PageRank,通过在相关联的两个顶点之间建立连边,然后再利用经验公式对图中每个顶点的权重进行迭代求解,直至收敛。收敛后的图中,每个顶点的权重即代表了这个顶点在模型中的重要性,换句话说权重大的顶点对应的词语在原文中也具有相对较高的重要性,更有可能是文档的关键词。相比于TF-IDF算法,TextRank不需要大量输入文档的训练,从仅单篇文档即可抽取出关键词。
Rake算法则相对于TextRank更为简单,不过思路大体相同,都是将文档建模为图。在Rake算法中,引入了词语的“度”这个概念,简单的说就是这个词语在整篇文档中非重复共现词的数量。通过计算每个词语的度,可以用经验公式“度除以出现频率”来计算出每个候选关键词的重要性得分,再根据关键词的得分来筛选出最终文档的关键词。
这些传统的算法经过了许许多多实验的检验,在各种各样的输入数据中均可以发挥不错的效果,但是在法律智能领域却遇到了不小的挑战。在法律智能领域中,输入数据会包含各种专业的法律术语,各种复杂而严谨的表达,使用传统的方法并不能理解这些术语与表达,不能很好的从这些语句中抽取出关键词。
关键词抽取在自然语言处理中,占据着重要的地位,它不仅作为信息归纳的工具,还辅助着一系列后继任务的进行,在法律智能领域解决好这一问题也是当务之急。










