\chapter{相关工作}
\label{cha:related}
当前人工智能与深度学习正在以势不可挡的冲击力影响着我们的生活,我们国家至关重要的司法领域并没有选择回避,而是采取了一个积极的姿态主动拥抱技术,其目的就是通过人工智能与法官审判制度的融合来促进司法改革。法律结合人工智能已经成为不可阻挡的趋势,法律智能这一领域也受到了学术界、商业界的广泛关注,与法律智能有关的任务也层出不穷。但是,由于技术发展时间短,许多任务并未达到令人满意的效果。这也导致了目前法律智能产品虽多,却大都无法满足实际使用的需求。


\section{学术研究现状}

法律智能领域的研究已经持续开展了几十年,不同的学者从不同的角度提出了很多不同的法律智能任务。其中研究最为广泛的就要属判决预测任务,在该任务中,算法模型需要通过阅读相关案情描述来给出判决结果。早在上个世纪就有学者尝试利用概率统计的算法来给出案件判决结果的概率分布\cite{kort1957predicting,ulmer1963quantitative,segal1984predicting}。随着机器学习算法在本实际初的迅速发展,许多学者将这样一个任务抽象为文本分类任务,并通过手动抽取文本的浅层次特征来抽取文本信息,并运用传统机器学习分类算法(朴素贝叶斯分类算法、决策树分类、支持向量机算法)来进行预测\cite{liu2006exploring,lin2012exploiting,aletras2016predicting}。但是这种做法需要耗费大量的精力总结文章中的相关的固定的规律,不具有可扩展性,因此并没有得到很好的运用。到了近几年,深度学习技术高速发展,许多自然语言处理领域(NLP)中的问题都得到了很好的解决。于是,许多学者又开始运用神经网络来进行预测罪名。\citet{luo2017learning}运用NLP中常有的注意力机制,在预测过程中引入法条内容,提升预测效果。\citet{hu2018few}引入了10个法学要素来解决案件罪名分布不均匀的问题。这些模型在其测试集上取得了较好的效果,但是由于这些任务都对实际运用场景做了简化,并未考虑法官判决时碰到的实际问题,因此这些模型在实际应用中仍然无法满足需求。

% 法律智能领域的研究已经持续开展了几十年,早在上个世纪就有许多学者尝试用概率统计算法来对案件进行分析[1][2],通过数学模型来预测案件的判决结果。作为法律智能的基础任务,判决预测这一任务自此也受到了广泛关注,随着本世纪初机器学习算法的发展,学者们将这一任务归结为文本分类任务,并尝试通过人工抽取特征并利用传统机器学习算法(朴素贝叶斯分类、SVM算法)[3][4]预测判决结果,到了近两年,学者又将深度学习技术运用其中[5]。这些模型虽然在测试集中取得了较好的效果,但是因为对问题的简化以及未考虑到判案时的实际因素,均无法应用到实际生活中。

% 与此同时,法律智能上还有许多其他问题也受到学者们的广泛关注。例如

与此同时,受到计算机学者、法学学者共同关注的法律智能领域许多其他任务也被纷纷提出。希望能够帮助解决我们生活中的种种法学问题。例如给定案情描述判断相关法条,与判决结果预测类似,学者也尝试使用文本分类建模,利用神经网络强大的抽取特征能力,寻找最相关的法条。不仅如此,人们还尝试利用人为定义的规则构建一套系统来判断法律问题的正误,但是此类方法缺乏可拓展性,且需要耗费大量的人力物力。


总而言之,目前法律智能领域受到广大自然语言处理领域的学者的欢迎,然而因为计算机学者法律知识的相对缺乏,以及学术研究上对问题的简化,目前学术领域研究虽然取得了很大的进展,但是离技术的落地与实际应用仍有一段很长的距离。

通过长期的调研与讨论,我们团队提出了一个新的模型,利用了机器学习中的多任务学习机制,通过多个子任务的联合训练提升了各个任务的效果;同时,考虑到法官的判案经过,我们提出了一个完整的模拟法官真实判案过程的模型,超过了之前提出的baseline模型,达到了真实可用的效果。


\section{商业产品现状}
目前市面上涌现出了许多的与法律服务有关的工具,根据团队调研结果,市场上的产品可以大致分成两类:以搜索引擎为主体的法律文书检索系统、以检索的技术为主的问答系统。接下来我们将分别介绍这样两类系统及其代表,并简要阐述其缺点。

文书检索系统是市面上出现最早的、目前应用也最广泛的法律应用。自从国家启动司法案例公开化之后,裁判文书网 便成为了拥有文书数量最多的官方平台,但是由于其检索速度慢、检索结果不全而没有成为一个使用广泛的工具。随后,许多其他检索工具(如无讼 、北大法宝 等)应运而生,相对于裁判文书网,这些网站增加了法律法条、指导案例的检索功能,但由于其采用的仍是基于词语匹配的检索算法,这些检索工具依旧无法从语义角度进行内容的检索,这样一个缺点在进行长文本检索时尤其突出。我们以北大法宝检索为例,在检索转化型抢劫的一般情形——“偷窃被发现,用暴力导致被害人受伤”时,北大法宝并未返回相关结果。因此,目前市场上的搜索引擎以词语匹配为技术手段,无法很好的服务于大众。

另一类市场上较多的法律产品是以检索技术为主的问答系统。问答是自然语言处理中的热门问题,但因为效果原因并未有很好的应用,而现有产品将问题进一步简化。例如京东推出的一款名为“法咚咚”的问答产品,通过将问题限定领域,并利用现有的检索技术,在用户输入问题时,通过词语匹配技术来推荐相关的法律依据。技术的限制也让这样一个产品有了许多限制:无法分析实际情景、使用者需要有法律背景。同期搜狗提出的“搜狗律师”也面临着同样的问题。\\

综上所述,目前市面上的法律产品并没有使用人工智能领域的前沿技术,而其服务的受众群体也是极其有限的,无法律知识背景的人在使用这些工具时会遇到很大的困难。

