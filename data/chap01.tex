\chapter{项目背景}
% \label{cha:intro}
\label{cha:background}


\section{社会背景}
法律是治国之重器,表明了国家权力的运行和国家意志的体现。法律在规范个人、保障社会秩序方面起到了非常重要的作用,同时法律也是我们每一个人维护自身权益的保证。因此,法律对于国家、社会、个人都是相当重要的。自新中国成立以来,全国上下各个司法机构、学术领域产生了大量的法律文书、法学期刊等文件。这些法学文件一直以来被法学领域的学者、法官、律师重视,成为了他们工作中非常重要的一个部分。自国家全面推行依法治国以来,法律在我们日常生活中扮演着越来越重要的角色。但是,当人们在生活中遇到法律问题时,却常常因为缺乏专业知识而不知道如何进行维权。同时,法学领域的学者、工作人员在面对浩如烟海的法律文本时,也常常因为缺乏好的检索工具和阅读系统,而需要花费大量的时间来查找相关材料,大大降低了工作效率。目前我国司法建设面临着严重的考验。

\textbf{案多人少形势严峻}。近几年来,中国每天发生的法律案件日益增加,据统计,2018年人民法院受理的案件高达2800万件,创下历史新高。每年案件数量依旧以14.5\%的速度进行增长,然而中国法官数量相比2012年却减少了40\%,这带给众多一线法官巨大的工作压力。案多人少的严峻局面给国家的司法公平公正提出了巨大的挑战。据数据统计,2018年平均每个法官一年审理案件高达233件,是2008年的\textbf{4倍}。然而法律领域对人才的严苛性让法官数量的增加受到很大的限制。

\textbf{法律服务需求缺口巨大。}目前,我国发生的案件中有80\%无法得到专业律师代理。与此相反,在美国人均律师数量是中国的15.6倍。同时我国律师地区分布也具有极大的差异性,80\%的律师集中在中国20\%的城市,这也就表明偏远地区人民在获取法律服务时更是难上加难。这些数据表明,法律服务在中国具有庞大的市场规模,但是由于人才供应不足,大多数人在遇到法律问题时,有求而无应。

\textbf{大量司法数据的累积。}自从国家立法以来,全国上下各级法院产生了大量的司法文件。得益于司法公开化进程推进,国家已经公开了超过6千万的法律案件文书、数千部法律与司法解释,同时,在众多的法学学者努力下创办了几百种权威的法学期刊。这为人工智能介入司法领域提供了数据基础。

\textbf{技术的高速发展。}目前,技术的创新给我们的生活带来了许许多多的改变,其中表现最突出的便是人工智能与大数据的技术。伴随着深度学习的发展,人工智能中的许多领域在近几年来取得了很大的进步。自然语言处理技术便是其中的代表,在该领域中文本分类、知识图谱、信息抽取、机器翻译等任务上,现有模型纷纷打破了前人的纪录。技术的创新带来的是生活的改变,而这样的改变也逐渐渗透到法学领域。2018年,司法部提出“加快‘数字法治、智慧司法’建设”的口号 ,这也让法学学者也越发注重人工智能在司法领域的应用。


\section{商业背景}

目前市面上涌现出了许多的与法律服务有关的工具,根据团队调研结果,市场上的产品可以大致分成两类:以搜索引擎为主体的法律文书检索系统、以检索的技术为主的问答系统。接下来我们将分别介绍这样两类系统及其代表,并简要阐述其缺点。

文书检索系统是市面上出现最早的、目前应用也最广泛的法律应用。自从国家启动司法案例公开化之后,裁判文书网 便成为了拥有文书数量最多的官方平台,但是由于其检索速度慢、检索结果不全而没有成为一个使用广泛的工具。随后,许多其他检索工具(如无讼 、北大法宝 等)应运而生,相对于裁判文书网,这些网站增加了法律法条、指导案例的检索功能,但由于其采用的仍是基于词语匹配的检索算法,这些检索工具依旧无法从语义角度进行内容的检索,这样一个缺点在进行长文本检索时尤其突出。我们以北大法宝检索为例,在检索转化型抢劫的一般情形——“偷窃被发现,用暴力导致被害人受伤”时,北大法宝并未返回相关结果。因此,目前市场上的搜索引擎以词语匹配为技术手段,无法很好的服务于大众。

另一类市场上较多的法律产品是以检索技术为主的问答系统。问答是自然语言处理中的热门问题,但因为效果原因并未有很好的应用,而现有产品将问题进一步简化。例如京东推出的一款名为“法咚咚”的问答产品,通过将问题限定领域,并利用现有的检索技术,在用户输入问题时,通过词语匹配技术来推荐相关的法律依据。技术的限制也让这样一个产品有了许多限制:无法分析实际情景、使用者需要有法律背景。同期搜狗提出的“搜狗律师”也面临着同样的问题。% 随着法律智能受到社会的广泛关注,各大有关公司也都推出了相应的产品。其中最具代表性的法律工具网站有:中国裁判文书网、北大法宝、法信网、法咚咚、搜狗法律等。这些产品大多以基于词匹配的案件文书检索为核心。受限于统计方法的检索能力,这些产品大多面向法律专业人士,旨在为他们提供基础的法律服务。
\begin{table}[]
\begin{tabular}{c|ccccc}
\hline
评价指标             & 语义检索       & 关键词标签      & 法学要素       & 案情分析预测     & 长文本检索      \\ \hline
裁判文书网            & ×          & ×          & ×          & ×          & ×          \\
北大法宝             & ×          & √          & ×          & ×          & ×          \\
威科先行             & ×          & ×          & ×          & ×          & ×          \\
无讼               & ×          & √          & ×          & ×          & ×          \\ \hline
\textbf{JudgeAI} & \textbf{√} & \textbf{√} & \textbf{√} & \textbf{√} & \textbf{√} \\
\hline
\end{tabular}
\caption{JudgeAI与市面上常用产品对比}
\label{background:compare}
\end{table}

\section{项目难点}
深度神经网络在图像分类任务上的成功,使得人工智能有了长足进步。许多工作已经证明,深度学习在自然语言处理领域同样


在这样的契机之下,团队构建了一个案情分析平台——JudgeAI,尝试将自然语言技术运用到法学领域的各个问题之上,旨在利用自然语言处理的技术的方法来解决实际法律问题,为大家的工作、生活提供便利。我们实现了对案情的全方面多角度的分析,在输入一段文本形式案情描述之后,我们平台将提供:案件的案由/罪名预测、判案要素预测、相关法条预测、相似案件检索、关键词提取和相关问题推荐的功能。以此让用户能够尽快捕捉到文本中的有用信息,并作出相应的判断,从而达到辅助判决的作用。相比于已有的产品、算法,我们团队在这些任务上提升了对长文本语义的理解,超过以往模型,达到了更高的预测精度。




