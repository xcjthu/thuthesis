\chapter{项目背景}
% \label{cha:intro}
\label{cha:background}


\section{社会背景}
法律是治国之重器,法律表明了国家权力的运行和国家意志的体现。法律在规范个人、保障社会秩序方面起到了非常重要的作用,同时法律也是我们每一个人维护自身权益的保证。因此,法律对于国家、社会、个人都是相当重要的。自新中国成立以来,全国上下各个司法机构、学术领域产生了大量的法律文书、法学期刊等文件。这些法学文件一直以来被法学领域的学者、法官、律师重视,成为了他们工作中非常重要的一个部分。自国家全面推行依法治国以来,法律在我们日常生活中扮演着越来越重要的角色。但是,当人们在生活中遇到法律问题时,却常常因为缺乏专业知识而不知道如何进行维权。同时,法学领域的学者、工作人员在面对浩如烟海的法律文本时,也常常因为缺乏好的检索工具和阅读系统,而需要花费大量的时间来查找相关材料,大大降低了工作效率。

据统计,近几年来,中国每天发生的法律案件日益增加,2018年人民法院受理的案件高达2800万件,创下历史新高 ;全国共有12万法官,平均每个法官一年审理案件233件;全国共拥有36.5万的职业律师,一年代理的诉讼案件达到465万件,却不足一年案件总数的20\%。上述数据显示,法律服务在中国有着庞大的市场规模。但是由于法律人才的供应不足,市场上对法律服务的需求始终没有办法得到满足。 

在另一方面,技术的创新给我们的生活带来了许许多多的改变,其中表现最突出的便是人工智能与大数据的技术。伴随着深度学习的发展,人工智能中的许多领域在近几年来取得了很大的进步。自然语言处理技术便是其中的代表,在该领域中文本分类、知识图谱、信息抽取、机器翻译等任务上,现有模型纷纷打破了前人的种种记录。技术的创新带来的是生活的改变,而这样的改变也逐渐渗透到法学领域。2018年,司法部提出“加快‘数字法治、智慧司法’建设”的口号 ,这也让法学学者也越发注重人工智能在司法领域的应用。


\section{商业背景}
随着法律智能受到社会的广泛关注,各大有关公司也都推出了相应的产品。其中最具代表性的法律工具网站有:中国裁判文书网、北大法宝、法信网、法咚咚、搜狗法律等。这些产品大多以基于词匹配的案件文书检索为核心。受限于统计方法的检索能力,这些产品大多面向法律专业人士,旨在为他们提供基础的法律服务。


在这样的契机之下,团队构建了一个案情分析平台——JudgeAI,尝试将自然语言技术运用到法学领域的各个问题之上,旨在利用自然语言处理的技术的方法来解决实际法律问题,为大家的工作、生活提供便利。我们实现了对案情的全方面多角度的分析,在输入一段文本形式案情描述之后,我们平台将提供:案件的案由/罪名预测、判案要素预测、相关法条预测、相似案件检索、关键词提取和相关问题推荐的功能。以此让用户能够尽快捕捉到文本中的有用信息,并作出相应的判断,从而达到辅助判决的作用。相比于已有的产品、算法,我们团队在这些任务上提升了对长文本语义的理解,超过以往模型,达到了更高的预测精度。


\section{项目难点}





