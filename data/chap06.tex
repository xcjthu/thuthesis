\chapter{实验结果}
\label{cha:result}

我们针对性的在案由/罪名预测、相关法条预测、刑期预测、关键词抽取任务上,开展了多个不同的实验。实验结果表明,我们提出的模型超过了以往对应的模型,实现了state-of-the-art的效果。

\section{判决预测模块}
\subsection{实验设置}
为了评估提出的判决预测模块,我们在中国刑事案件的三个大型数据集上进行了一系列实验。我们选择三个具有代表性的判断预测子任务进行比较,包括法律条款,罪名和刑期。

针对于判决预测模块提出的多任务学习模型,我们在多个相关数据集上进行了相应的实验。我们根据北大法宝、中国裁判文书网、中国司法挑战赛三个数据来源,构建了相应的三个判决预测数据集,并按照 $8:2$ 的比例划分训练集与测试集。

由于案例文件是用中文书写的,文字之间没有空格,我们采用THULAC\cite{sun2016thulac}进行分词。 之后,我们采用 CBOW模型\cite{mikolov2013distributed}来在上述数据集上预先训练词向量,词向量维数设置为 $200$。对于所有模型,我们将事实表示和特定于任务的表示向量大小设置为 $256$。同时,我们将最大句子长度设置为128个单词,最大文档长度为 $32$ 个句子。对于训练,我们使用了Adam优化器,并设置学习率为 $10-3$,丢失(dropout)概率为 $0.5$。 我们还为所有模型将训练批量大小设置为 $128$。 我们训练每个模型 $16$轮,并在测试集上评估最终模型。

我们使用准确度(Accracy, Acc),宏观精度(Macro-Precision, MP),宏观召回(Marco-Recall, MR)和宏观F1(Marco-F1, F1)作为评估指标。 这里,宏精度/召回率/F1是通过平均每个类别的精度/召回率/F1来计算公式如下。
\begin{itemize}
	\item 准确率(Accuracy)。顾名思义,就是所有的预测正确(正类负类)的占总的比重。
		\begin{equation}
			\text {Accuracy}=\frac{TP+TN}{TP+TN+FP+FN}
		\end{equation}
	\item 精确率(Precision),查准率。即正确预测为正的占全部预测为正的比例。即,真正正确的占所有预测为正的比例。
		\begin{equation}
			\text { Precision }=\frac{TP}{TP+FP}
		\end{equation}
	\item 召回率(Recall),查全率。即正确预测为正的占全部实际为正的比例。即,真正正确的占所有实际为正的比例。
		\begin{equation}
			Recall=\frac{TP}{TP+FN}
		\end{equation}
	\item F1值。F1值为算数平均数除以几何平均数,且越大越好,将Precision和Recall的上述公式带入会发现,当F1值小时,True Positive相对增加,而false相对减少,即Precision和Recall都相对增加,即F1对Precision和Recall都进行了加权。
		\begin{equation}
			\frac{2}{F_{1}}=\frac{1}{\text { Precision }}+\frac{1}{\text { Recall }}
		\end{equation}
\end{itemize}

在这里,TP(True Positives)为预测结果正确的正类;TN(True Negatives)预测结果正确的负类;FP(False Positives)为预测结果错误的正类;FN(False Negatives)为预测结果错误的负类。

\begin{table}[ht]
\begin{tabular}{l|llll|llll|llll}
\hline
任务        & \multicolumn{4}{l|}{相关法条推荐} & \multicolumn{4}{l|}{案由/罪名预测} & \multicolumn{4}{l}{刑期预测} \\ \hline
评价指标      & Acc   & MP    & MR   & F1   & Acc   & MP    & MR   & F1   & Acc  & MP   & MR   & F1   \\ \hline
TFIDF+SVM & 82.4  & 45.5  & 26.7 & 30.2 & 82.2  & 47.4  & 27.9 & 31.3 & 48.5 & 36   & 16.7 & 16.5 \\
CNN       & 92.5  & 46.9  & 38.4 & 40.0 & 92.3  & 41.2  & 32.3 & 33.7 & 57.4 & 35.6 & 22.2 & 22.7 \\
Fact-law  & 93.5  & 50.9  & 45.6 & 45.9 & 93.4  & 47.2  & 41.4 & 41.5 & 56.3 & 31.3 & 26.4 & 26.7 \\
Pipeline  & 93.7  & 51.9  & 44.1 & 44.9 & 93.6  & 45.5  & 39.1 & 39.3 & 58.2 & 38.2 & 24.9 & 26.8 \\ \hline
\textbf{Ours}      & \textbf{94.4}  & \textbf{53.9}  & \textbf{47.3} & \textbf{48.2} & \textbf{94.9}  & \textbf{53.9}  & \textbf{48.2} & \textbf{49.1} & \textbf{58.5} & \textbf{40.2} & \textbf{32.9} & \textbf{32.8} \\ \hline
\end{tabular}
\caption{模型在裁判文书网数据集上效果}
\label{table: result_judge_cjo}
\end{table}


\begin{table}[ht]
\begin{tabular}{l|llll|llll|llll}
\hline
任务        & \multicolumn{4}{l|}{相关法条推荐} & \multicolumn{4}{l|}{案由/罪名预测} & \multicolumn{4}{l}{刑期预测} \\
\hline
评价指标      & Acc    & MP   & MR   & F1   & Acc     & MP   & MR   & F1   & Acc  & MP   & MR   & F1   \\ \hline
TFIDF+SVM & 80.9   & 51.3 & 32.6 & 36.4 & 81.0    & 53.4 & 35.4 & 38.7 & 45.3 & 30.4 & 17.4 & 17.2 \\
CNN       & 93.1   & 64.3 & 52.6 & 54.3 & 93.3    & 61.9 & 49.3 & 51.1 & 57.6 & 24.1 & 23.1 & 23.3 \\
Fact-law  & 93.9   & 68.1 & 63.4 & 63.5 & 94.2    & 65.8 & 58.5 & 58.7 & 55.7 & 27.7 & 27.4 & 26.5 \\
Pipeline  & 94.4   & 69.6 & 61.0 & 62.2 & 94.3    & 65.1 & 56.2 & 57.2 & \textbf{58.2} & 36.2 & 26.4 & 27.1 \\ \hline
Ours      & \textbf{95.4}   & \textbf{76.4} & \textbf{67.6} & \textbf{68.4} & \textbf{95.6}    & \textbf{75.9} & \textbf{69.6} & \textbf{70.9} & 57.8 & \textbf{38.9} & \textbf{32.1} & \textbf{31.8} \\ \hline
\end{tabular}
\caption{模型在北大法宝数据集上的效果}
\label{table: result_judge_pku}
\end{table}

为了比较,我们采用以下文本分类模型和判断预测方法作为基线模型:

\begin{itemize}
	\item TFIDF+SVM:这是文本分类最常用的传统分类算法,我们应用了词频与逆文档频率TFIDF抽取文档特征,并以此为基础采用了支持向量机SVM进行分类。
	\item CNN:我们采用了卷积神经网络[14]作为编码器,让机器在阅读文章后自动抽取出对分类有用的特征,再将其映射到各个罪名、法条、刑期上。为了对比,我们在改编码器上运用了基本的多任务学习机制,来与我们的模型进行对比。
	\item Fact-Law:是由北大团队自然语言处理团队在2017年提出的罪名预测模型。模型通过引入法条描述,利用Attention机制来提升模型预测罪名的效果。
	\item Pipeline:这是一个基本的多任务学习的模型,该模型为不同的子任务训练不同的编码器,通过在输出层融合不同任务的信息,捕获任务之间的依赖关系。
\end{itemize}



\subsection{实验结果与分析}
由表中数据可知,我们的模型在相关法条推荐、案由/罪名预测、刑期预测三个任务上超过了之前的baseline,取得了很好的效果。多个数据集上的实验体现了我们提出模型的鲁棒性与有效性。

\begin{table}[]
\begin{tabular}{l|llll|llll|llll}
\hline
任务        & \multicolumn{4}{l}{相关法条推荐} & \multicolumn{4}{l|}{案由/罪名预测} & \multicolumn{4}{l}{刑期预测}  \\
\hline
评价指标      & Acc   & MP   & MR   & F1   & Acc   & MP    & MR   & F1   & Acc  & MP   & MR   & F1   \\ \hline
TFIDF+SVM & 60.1  & 54.9 & 45.3 & 46.3 & 59.2  & 53.9  & 45.0 & 45.7 & 28.4 & 22.9 & 20.0 & 18.1 \\
CNN       & 81.4  & 74.4 & 64.1 & 65.7 & 80.7  & 77.3  & 65.5 & 67.2 & 28.8 & 34.7 & 27.8 & 28.6 \\
Fact-law  & 70.9  & 64.8 & 63.6 & 59.1 & 68.7  & 66.1  & 65.3 & 60.1 & 36.5 & 29.9 & 27.6 & 27.1 \\
Pipeline  & 84.7  & 80.7 & 68.6 & 70.8 & 83.6  & 81.6  & 70.0 & 72.1 & \textbf{40.0}   & \textbf{37.4} & 32.0 & 31.6 \\ \hline
Ours      & \textbf{86.3}  & \textbf{81.9} & \textbf{71.1} & \textbf{73.4} & \textbf{85.7}  & \textbf{83.4}  & \textbf{76.0}   & \textbf{78.3} & 38.3 & 36.1 & \textbf{33.1} & \textbf{32.1} \\ \hline
\end{tabular}
\caption{模型在CAIL数据集上效果}
\label{table: result_judge_cail}
\end{table}

同时,由于考虑了多个任务之间的相互依赖关系,我们的模型解决了预测结果互相矛盾的问题。例如模型在相关法条预测结果为涉及“盗窃罪”的法条,而在案由/罪名预测时却将被告判断为“故意杀人罪”,这样的预测结果矛盾现象在其他模型中非常常见。我们的模型很好的解决了这样一个问题,因此实现了优越的效果。即,实验结果表明:
\begin{itemize}
	\item 在大多数子任务和数据集上,团队提出的的判决预测模型显著优于其他基线模型。证明了我们提出的框架的有效性和可靠性。
	\item 此外,我们提出的模型明显优于典型的多任务学习模型,特别是在罪名预测和刑期预测方面。它验证了使用循环神经元对判决预测子任务建立依赖关系的合理性和重要性。
\end{itemize}


\section{关键词抽取模块}
\subsection{实验设置}
我们进行了一系列广泛的实验,以研究提出的模型在融合字信息与词语信息的有效性。此外,我们的目标是在不同的设置下根据经验比较基于单词和基于字母的中文关键词抽取。我们使用了标准精度(Precision),召回(Recall)和F1分数(F1)用作评估指标。

由于之前并没有法律领域的关键词数据集,我们采用了人工构造的方法。我们从大量的法律咨询语句与民事案件文书中描述争议焦点的语句中筛选待标注候选的语句。得到候选语句集合之后,我们利用THULAC工具进行中文分词,每一条数据经过三次标注来保证标注结果的可靠性。
同时,我们利用了以下传统的关键词抽取模型作为模型baseline进行对比。
\begin{table}[ht]
\centering
\begin{tabular}{llll}
\hline
       & Precision  & Recall & F1    \\ \hline
CRF        & 58.16  & 41.61 & 48.51 \\
BiLSTM-CRF & 62.45  & 61.46 & 63.47 \\ \hline
Ours       & \textbf{68.14}  & \textbf{67.83} & \textbf{67.99} \\ \hline
\end{tabular}
\caption{关键词抽取算法效果}
\end{table}

\begin{itemize}
	\item CRF:条件随机场是一个传统的基于概率统计的序列标注模型。第一步人工抽取特征,模型通过学习基于这些特征的关键词概率分布来进行训练。
	\item BiLSTM+CRF:双向LSTM与CRF的结合,是目前神经网络运用于序列标注的最主要的模型。利用神经网络抽取特征的能力,来避免人工抽取特征的不全面与高代价。
\end{itemize}

\subsection{实验结果与分析}

从实验结果可见,我们的模型在关键词提取任务上实现了大幅的提升。相比于传统的字级别的序列标注模型与词级别的序列标注模型,我们使用的模型综合了字级别的信息与词级别的信息,以此避免了词级别模型过度依赖分词效果与字级别模型无法捕捉到足够信息的缺点。



