\chapter{平台功能与成果概述}

\section{功能介绍}
平台围绕着输入的案情描述进行了全方面多角度的分析,我们实现以下几个功能:

1)	罪名/案由预测,输入一段案情后,模型通过获取其语义,给出相关的案由/罪名及相关概率;

2)	相关法条推荐,对案情进行分析后,模型将给出与案件有关的法律法规;

3)	刑期预测,对每一个刑事案件,模型都将给出可能的判罚刑期,我们将刑期合理的划分成几个区间,模型将给出被告判罚的刑期落入每一个区间的概率;

4)	关键词抽取,通过序列标注模型,模型为每一个输入案情给出相应的法学关键词;

5)	类案推荐,在输入案情描述之后,我们将在数据库中进行检索,并通过模型计算,给出在语义层面与输入案件相关的法律文书。


\section{成果概述}

通过长时间的努力,团队取得了许多阶段性成就。

1)	大规模的法律文书数据集,通过长时间分布式的数据抓取,我们最终获得了数千万的法律文书,其中包括刑法500余万份;

2)	关键词抽取数据集,为了实现关键词抽取模型,我们整理了10000万份待标注的法律相关句子,通过专业人士的人工标注,我们最终获得了这样一份高质量的关键词抽取数据集;

3)	判决预测模块算法的实现,提出了一个新的多任务学习模型,取得了很好的效果;

4)	关键词模块算法的实现,将目前自然语言处理领域最前沿的序列标注模型应用至关键词抽取任务上,通过特征工程等方法,对算法进行了改进;

5)	类案推荐搜索引擎的实现,提出两步走的检索算法,做到了语义相似性检索;

6)	平台搭建,将所有算法集成到demo平台上,形成一个可用的法律产品。
