\chapter{平台功能与成果概述}

\section{功能介绍}
平台围绕着输入的案情描述进行了全方面多角度的分析,我们实现以下几个功能:
\begin{enumerate}[1)]
	\item \textbf{罪名/案由预测}。作为每一个案件最基本的特征,案由的准确预测可以让大家清晰的捕捉到案件最核心的诉求。通常法院审查案件的第一步便是确定案件涉及的案由,并以此为基础,查询相应的法学资料、相关的案例,来做出最终的判决。在我们的平台中,我们同样以案由的预测作为分析案情的第一步。在输入一段案情后,算法利用了深度学习技术进行案件推理,来确定相关的案由/罪名并给出相应的概率。
	\item \textbf{相关法条推荐}。在大陆法系中,法律法规是法官在审判案件时的基本依据,所有案件的审判都需要做到有法可依、有法必依。每一个案件的判决结果的背后都需要有许多的法律法规来为其提供理论支持。但是,我们在日常生活中遇到法律问题时,常常因为无法找到合适的法律法规,而无法合理地保障我们自身的权益。因此,平台提供了相关法条推荐的功能,从理论的角度,为案情的分析提供合适的依据。
	\item \textbf{刑期预测}。刑事处罚分为主刑与附加刑两种,其中主刑有:管制、拘役、有期徒刑、无期徒刑和死刑。该功能就是预测刑事案件每一个
	对每一个刑事案件,模型都将给出可能的判罚刑期,我们将刑期合理的划分成几个区间,模型将给出被告判罚的刑期落入每一个区间的概率;
	\item \textbf{关键词抽取},通过序列标注模型,模型为每一个输入案情给出相应的法学关键词;
	\item \textbf{法律要素抽取}
	\item \textbf{事件抽取}
	\item \textbf{相关问题推荐}
	\item \textbf{类案推荐},在输入案情描述之后,我们将在数据库中进行检索,并通过模型计算,给出在语义层面与输入案件相关的法律文书。
\end{enumerate}


\section{成果概述}

通过长时间的努力,团队取得了许多阶段性成就。

\begin{enumerate}[1)]
	\item 大规模的法律文书数据集,通过长时间分布式的数据抓取,我们最终获得了数千万的法律文书,其中包括刑法500余万份;

	\item 关键词抽取数据集,为了实现关键词抽取模型,我们整理了10000万份待标注的法律相关句子,通过专业人士的人工标注,我们最终获得了这样一份高质量的关键词抽取数据集;

	\item 判决预测模块算法的实现,提出了一个新的多任务学习模型,取得了很好的效果;

	\item 关键词模块算法的实现,将目前自然语言处理领域最前沿的序列标注模型应用至关键词抽取任务上,通过特征工程等方法,对算法进行了改进;

	\item 类案推荐搜索引擎的实现,提出两步走的检索算法,做到了语义相似性检索;

	\item 平台搭建,将所有算法集成到demo平台上,形成一个可用的法律产品。
\end{enumerate}

